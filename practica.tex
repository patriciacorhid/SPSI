\documentclass{article}
\usepackage[left=1.8cm,right=3cm,top=2cm,bottom=2cm]{geometry} % page
% settings
\usepackage{multicol}
\usepackage{amsmath,amsthm,amssymb}
\usepackage{amsfonts}
\usepackage{dsfont}
\usepackage{upgreek}
\usepackage{parskip}
\usepackage[spanish]{babel}
\usepackage[doument]{ragged2e}

% Images
\usepackage{graphicx}
\usepackage{float}
\usepackage{subfigure} % subfiguras
\usepackage{caption}
\captionsetup[table]{labelformat=empty}
\captionsetup[figure]{labelformat=empty}

% Code
\usepackage{tikz}
\usetikzlibrary{automata,positioning}
\usepackage{pgfplots}
\usepackage{color}

\usepackage{listings}
\usepackage{xcolor}
\definecolor{gray}{rgb}{0.5,0.5,0.5}
\newcommand{\n}[1]{{\color{gray}#1}}
\lstset{numbers=left,numberstyle=\small\color{gray}}

\usepackage[bookmarks=true,
            bookmarksnumbered=false, % true means bookmarks in 
                                     % left window are numbered
            bookmarksopen=false,     % true means only level 1
                                     % are displayed.
            colorlinks=true,
            allcolors=blue,
            urlcolor=cyan]{hyperref}

\selectlanguage{spanish}
\usepackage[utf8]{inputenc}
\setlength{\parindent}{0mm}

\usepackage{xcolor}
\usepackage{listings}
\lstset{basicstyle=\ttfamily,
  showstringspaces=false,
  numbers=none
}

\begin{document}

\title{\Huge \textbf{Cifrado con curvas elípticas usando OpenSSL}} \author{Yabir
García Benchakhtir\\ David Cabezas Berrido\\ Patricia Córdoba Hidalgo
} \date{}
\maketitle

\section{Introducción}

OpenSSL es una herramienta muy completa con la que hemos trabajado en la
asignatura y que además nos da la facilidad para trabajar también con curvas
elípticas. En esta práctica vamos a realizar pasos análogos a lo que hemos visto
durante el curso pero aplicando la teoría de curvas elípticas. 

\section{Detalles a tener en cuenta}

Durante nuestros intentos de cifrar y descifrar utilizando ECC hemos encontrado
\href{https://stackoverflow.com/a/58942471/2588566}{este comentario} donde se
nos indica que no podemos realizar estos pasos directamente usando ECC. Usamos
lo que se ha llama \textit{Elliptic Curve Integrated Encryption Scheme}.

\section{Taller de criptografía}

En primer lugar comentamos que OpenSSL nos ofrece una serie de curvas ya
definidas en su utilidad para terminal. Tenemos la posibilidad de listar las
curvas disponibles utilizando:

\begin{lstlisting}[language=bash]
  openssl ecparam -list_curves
\end{lstlisting}

tras ejecutar esta instrucción se nos mostrará una lista con distintas curvas
conocidas. Para continuar con el taller vamos a elegir una de estas curvas,
en nuestro caso \textit{secp256k1} que es una curva famosa y bastante segura.

Para crear un archivo de configuración para la curva utilizamos 

\begin{lstlisting}[language=bash]
  openssl ecparam -name secp256k1 -out secp256k1.pem
\end{lstlisting}

que creará un achivo \textit{secp256k1.pem} con información sobre la curva. Si
ahora hacemos

\begin{lstlisting}[language=bash]
  openssl ecparam -in secp256k1.pem -text -param_enc explicit -noout
\end{lstlisting}

obtenemos información sobre los parametros elegidos como: 

\begin{itemize}
  \item Los parámetros $A$ y $B$ de la ecuación de la curva.
  \item El número primo elegido para el cuerpo sobre el que se construye la curva.
  \item El generador del punto cíclico, $G$.
  \item El orden del grupo $<G>$.
  \item El cofactor del grupo cíclico.
\end{itemize}

Una vez elegida la curva procedemos ahora a crear la clave privada. Para ello
utilizamos 

\label{genkey}
\begin{lstlisting}[language=bash]
  openssl ecparam -in secp256k1.pem -genkey -noout -out userS-privkey.pem
\end{lstlisting}

Con esta orden le indicamos que genere en un archivo \textit{userS-privkey.pem}
una clave privada partiendo del archivo de configuración para la curva que hemos
generado anteriormente. Si deseamos ver la información de la clave privada y la pública asociada que
hemos generado podemos usar 

\begin{lstlisting}[language=bash]
  openssl ec -in userS-privkey.pem -text -noout
\end{lstlisting}

Si queremos guardar la clave pública en un archivo tenemos que hacer 

\begin{lstlisting}[language=bash]
  openssl ec -in userS-privkey.pem -pubout -out userS-pubkey.pem
\end{lstlisting}

donde a partir de nuestra clave privada generamos un archivo de clave pública
con la clave pública generada. Para ver la información almacenada en el nuevo
archivo ejecutamos

\begin{lstlisting}[language=bash]
  openssl ec -in userS-pubkey.pem -pubin -text -noout
\end{lstlisting}

En este paso podemos comprobar  que el contenido coincide con la información que
obtuvimos al generar la clave privada.

Procedemos a firmar ahora un mensaje. Creamos un archivo para ello que hemos
llamado \textit{msg.txt} y procedemos a firmarlo utilizando la clave privada que
hemos generado con anterioridad.

\begin{lstlisting}[language=bash]
  openssl dgst -sign userS-privkey.pem -out msg.txt.sign msg.txt
\end{lstlisting}

Para realizar el cifrado creamos un secreto compartido generado a partir de nuestra clave
privada y la clave pública del receptor (suponiendo las claves generadas e intercambiadas). Esto lo hacemos por que OpenSSL no nos permite cifrar directamente
usando el archivo de la clave privada generada en los primeros pasos.

\begin{lstlisting}[language=bash]
  openssl pkeyutl -derive -inkey userS-privkey.pem \
    -peerkey userR-pubkey.pem -out secret.txt
\end{lstlisting}

Ciframos ahora utilizando el archivo \textit{secreto.txt} generado como clave

\begin{lstlisting}[language=bash]
  openssl enc -aes-256-cbc -md sha512 -pbkdf2 -iter 100000 -salt \
   -in msg.txt -out msg.txt.enc -pass file:secret.txt
\end{lstlisting}

además le hemos indicado que usamos \textit{sha512} como función de resumen y
\textit{aes-256-cbc} como sistema de cifrado.

Ahora el receptor del mensaje debe seguir los siguientes pasos. En
primer lugar crea el mismo secreto, pero esta vez usando su clave privada y la clave pública del emisor

\begin{lstlisting}[language=bash]
  openssl pkeyutl -derive -inkey userR-privkey.pem -peerkey \
    userS-pubkey.pem -out secret.txt
\end{lstlisting}

Para descrifrar se añade la opción \texttt{-d}

\begin{lstlisting}[language=bash]
  openssl enc -aes-256-cbc -md sha512 -pbkdf2 -iter 100000 -salt -d \
    -in msg.txt.enc -out msg.txt -pass file:secret.txt
\end{lstlisting}

Y por último para verificar la firma del documento usamos la clave pública del emisor y el documento de la firma sobre el documento que hemos
obtenido al descifrar.

\begin{lstlisting}[language=bash]
  openssl dgst -verify userS-pubkey.pem -signature msg.txt.sign msg.txt
\end{lstlisting}


\end{document}
